\documentclass[nonacm]{acmart}


%% end of the preamble, start of the body of the document source.
\begin{document}

%% Hide ACM Reference Foramt
\settopmatter{printacmref=false}


\title{Scalable but Wasteful: Current State of Replication in the Cloud}
\author{Sebastian Schneider, 03783507}
\email{go69coq@tum.de}


\maketitle

\section{Summary}
The write-up must be \underline{at least 4 pages long}, with  \underline{a maximum of 6 pages}.
The summary section should be approximately \underline{two pages} in length and include the key elements from the paper. It should resemble an abstract but provide more detailed information.

- modern consensus protocols  increase througput by eliminating bottlenecks, lowering efficiency in the process.
    - they increase performance by utilizing resources, that would otherwise be idle.
        - how they do that: Leaderless, shifting most of the work
- first test
    - setup
    - pure througput analysis
- efficiency metric
    - cpu usage of first test
    - new metric 
    - results
- cloud-native systems share resources
    - don't use dedicated resources
    - use multiple independent instances on the same machine
    - why efficiency even matters
- second test
    - setup
    - results
    - new metric predicted relative performance of protocols in shared environment
- call for stopping development of protocols optimized for dedicated resources and start development of protocols specifically meant for cloud usage

\section{Related Work}
The related work section must incorporate \underline{at least one related paper} that has not been addressed in the document.
This section should compare the primary idea of the paper with those of the related works. A good starting point is to explore papers that cite the given paper.
The length of this section should be \underline{about half a page}.

\section{Critique}
Critique the paper by identifying \textbf{three strong points} and \textbf{three weaknesses}.
The critique section should be \underline{at least one page long}.

\subsection{Strong Points}
% TODO
- Metric design

- stretched on 2:
Focus on shifting research away from outdated technologies towards already relevant and growing topics
- good highlighting of current problems    
    - very direct criticism of current methods
- good highlighting of promising directions


\subsection{Weak Points}
% TODO
- very limited scope: only suggestion for further research
    - stays theoretical even in tests

- repetitive

- 

\section{Suggestions for Improvement}
Propose ways to \textbf{enhance the paper's core idea} or outline \textbf{future extensions} that are not covered in the paper.
It is also acceptable to discuss \textbf{potential use cases} for the proposed idea or system that the paper does not mention.
Alternatively, consider suggesting \textbf{implementation approaches} to validate the proposed scheme and either prove or disprove its effectiveness (if you believe it may not be effective).
This section should be \underline{at least half a page long}.
% TODO



\end{document}